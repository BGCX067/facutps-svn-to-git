% This is the manual for the LaTeX hyperref package.
%
% Copyright (C) 1998, 2003 Sebastian Rahtz.
%
% Permission is granted to copy, distribute and/or modify this document
% under the terms of the GNU Free Documentation License, Version 1.1 or
% any later version published by the Free Software Foundation; with no
% Invariant Sections, with no Front-Cover Texts, and with no Back-Cover
% Texts.  A copy of the license is included in the section entitled
% ``GNU Free Documentation License.''
%
% Manual updates:
% * Steve Peter and Karl Berry, 7/03.
% * Heiko Oberdiek, 2006-2008.
%

\RequirePackage{ifpdf}
\ifpdf % We are running pdfTeX in pdf mode
\documentclass[pdftex]{article}
\else
\documentclass{article}
\fi

\usepackage{etex}% for \eTeX

% Page layout.
\advance\textwidth by 1.1in
\advance\oddsidemargin by -.55in
\advance\evensidemargin by -.55in
%
\advance\textheight by 1in
\advance\topmargin by -.5in
\advance\footskip by -.5in
%
\pagestyle{headings}
%
% Avoid some overfull boxes.
\emergencystretch=.1\hsize
\hbadness = 3000

% these are from lshort.sty, but lshort.sty pulls in so many other
% packages it seems cleaner to just include them here.
%
\newcommand{\bs}{\symbol{'134}}%Print backslash
\newcommand{\ci}[1]{\texttt{\bs#1}}

\makeatletter
\@ifpackageloaded{tex4ht}{%
  % separate definition for HTML case to avoid
  % nasty borders with double horizontal lines with
  % large gaps.
  \newsavebox{\cmdsyntaxbox}%
  \newenvironment{cmdsyntax}{%
    \par
    % \small
    \addvspace{3.2ex plus 0.8ex minus 0.2ex}%
    \vskip -\parskip
    \noindent
    \begin{lrbox}{\cmdsyntaxbox}%
      \begin{tabular}{l}%
        \rule{0pt}{1em}%
        \ignorespaces
  }{%
      \end{tabular}%
    \end{lrbox}%
    \fbox{\usebox{\cmdsyntaxbox}}%
    \par
    \nopagebreak
    \addvspace{3.2ex plus 0.8ex minus 0.2ex}%
    \vskip -\parskip
  }%
}{%
  \newenvironment{cmdsyntax}{%
    \par
    \small
    \addvspace{3.2ex plus 0.8ex minus 0.2ex}%
    \vskip -\parskip
    \noindent
    \begin{tabular}{|l|}%
      \hline
      \rule{0pt}{1em}%
      \ignorespaces
  }{%
      \\%
      \hline
    \end{tabular}%
    \par
    \nopagebreak
    \addvspace{3.2ex plus 0.8ex minus 0.2ex}%
    \vskip -\parskip
  }%
}
\makeatother

\usepackage{array,longtable}
\usepackage[T1]{fontenc}
\usepackage{lmodern}

\newcommand*{\Quote}[1]{\textquotedblleft#1\textquotedblright}

\def\Hanh{H\`an Th\^e\llap{\raise 0.5ex\hbox{\'{}}} Th\`anh}

\ifpdf
  \usepackage[%
    pdftex,%
    colorlinks,%
    hyperindex,%
    plainpages=false,%
    bookmarksopen,%
    bookmarksnumbered,
    pdfusetitle,%
  ]{hyperref}
  %%?? \def\pdfBorderAttrs{/Border [0 0 0] } % No border arround Links
  \usepackage{thumbpdf}
\else
  \usepackage{hyperref}
\fi

\title{Hypertext marks in \LaTeX: a manual for \textsf{hyperref}}
\author{Sebastian Rahtz \and  Heiko Oberdiek}
\date{October 2008}

\begin{document}

% comes out too close to the toc, and we know it's page one anyway.
\thispagestyle{empty}
\maketitle
\tableofcontents
\setcounter{tocdepth}{2}% for bookmark levels

\section{Introduction}

The package derives from, and builds on, the work of the Hyper\TeX\
project, described at \url{http://xxx.lanl.gov/hypertex/}. It extends
the functionality of all the \LaTeX\ cross-referencing commands
(including the table of contents, bibliographies etc) to produce
\verb|\special| commands which a driver can turn into hypertext links;
it also provides new commands to allow the user to write \emph{ad hoc}
hypertext links, including those to external documents and URLs.

This manual provides a brief overview of the \textsf{hyperref}
package. For more details, you should read the additional documentation
distributed with the package, as well as the complete documentation by
processing \texttt{hyperref.dtx}. You should also read the chapter on
\textsf{hyperref} in \textit{The \LaTeX\ Web Companion}, where you will
find additional examples.

The Hyper\TeX\ specification\footnote{This is borrowed from an article
by Arthur Smith.} says that conformant viewers/translators must
recognize the following set of \verb|\special| constructs:

\begin{description}
\item[href:] \verb|html:<a href = "href_string">|
\item[name:] \verb|html:<a name = "name_string">|
\item[end:] \verb|html:</a>|
\item[image:] \verb|html:<img src = "href_string">|
\item[base\_name:] \verb|html:<base href = "href_string">|
\end{description}

The \emph{href}, \emph{name} and \emph{end} commands are used to do the
basic hypertext operations of establishing links between sections of
documents. The \emph{image} command is intended (as with current HTML
viewers) to place an image of arbitrary graphical format on the page in
the current location. The \emph{base\_name} command is be used to
communicate to the DVI viewer the full (URL) location of the current
document so that files specified by relative URL's may be retrieved
correctly.

The \emph{href} and \emph{name} commands must be paired with an
\emph{end} command later in the \TeX\ file---the \TeX\ commands between
the two ends of a pair form an \emph{anchor} in the document. In the
case of an \emph{href} command, the \emph{anchor} is to be highlighted
in the \emph{DVI viewer}, and when clicked on will cause the scene to
shift to the destination specified by \emph{href\_string}. The
\emph{anchor} associated with a name command represents a possible
location to which other hypertext links may refer, either as local
references (of the form \verb|href="#name_string"| with the
\emph{name\_string} identical to the one in the name command) or as part
of a URL (of the form \emph{URL\#name\_string}). Here
\emph{href\_string} is a valid URL or local identifier, while
\emph{name\_string} could be any string at all: the only caveat is that
`$\verb|"|$' characters should be escaped with a backslash
($\backslash$), and if it looks like a URL name it may cause problems.

However, the drivers intended to produce \emph{only} PDF use literal
PostScript or PDF \verb|\special| commands. The commands are defined in
configuration files for different drivers, selected by package options;
at present, the following drivers are supported:

\begin{description}
\item[hypertex] DVI processors conforming to the Hyper\TeX\ guidelines (i.e.\ \textsf{xdvi}, \textsf{dvips} (with
the \textsf{-z} option), \textsf{OzTeX}, and \textsf{Textures})
\item[dvips] produces \verb|\special| commands tailored for \textsf{dvips}
\item[dvipsone] produces \verb|\special| commands tailored for \textsf{dvipsone}
\item[ps2pdf] a special case of output suitable for processing by earlier versions of Ghost\-script's
PDF writer; this is basically the same as that for \textsf{dvips}, but a few variations remained before version 5.21
\item[tex4ht] produces \verb|\special| commands for use with \textsf{\TeX4ht}
\item[pdftex] pdf\TeX, \Hanh{}'s \TeX{} variant that writes PDF directly
\item[dvipdfm] produces \verb|\special| commands for Mark Wicks' DVI to PDF driver \textsf{dvipdfm}
\item[dvipdfmx] produces \verb|\special| commands for driver
     \textsf{dvipdfmx}, a successor of \textsf{dvipdfm}
\item[dviwindo] produces \verb|\special| commands that Y\&Y's Windows previewer interprets as hypertext jumps within the previewer
\item[vtex] produces \verb|\special| commands that MicroPress' HTML and
     PDF-producing \TeX\ variants interpret as hypertext jumps within the
     previewer
\item[textures] produces \verb|\special| commands that \textsf{Textures} interprets as hypertext jumps within the previewer
\item[xetex] produces \verb|\special| commands for Xe\TeX{}
\end{description}

Output from \textsf{dvips} or \textsf{dvipsone} must be processed using
Acrobat Distiller to obtain a PDF file.\footnote{Make sure you turn off
the partial font downloading supported by \textsf{dvips} and
\textsf{dvipsone} in favor of Distiller's own system.} The result is
generally preferable to that produced by using the \textsf{hypertex}
driver, and then processing with \textsf{dvips -z}, but the DVI file is
not portable. The main advantage of using the Hyper\TeX\ \ci{special}
commands is that you can also use the document in hypertext DVI viewers,
such as \textsf{xdvi}.

\section{Implicit behavior}

This package can be used with more or less any normal \LaTeX\ document
by specifying in the document preamble

\begin{verbatim}
\usepackage{hyperref}
\end{verbatim}

Make sure it comes \emph{last} of your loaded packages, to give it a
fighting chance of not being over-written, since its job is to redefine
many \LaTeX\ commands. Hopefully you will find that all cross-references
work correctly as hypertext. For example, \ci{section} commands will
produce a bookmark and a link, whereas \ci{section*} commands will only
show links when paired with a corresponding \ci{addcontentsline}
command.

In addition, the \texttt{hyperindex} option (see below) attempts to make
items in the index by hyperlinked back to the text, and the option
\texttt{backref} inserts extra `back' links into the bibliography for
each entry. Other options control the appearance of links, and give
extra control over PDF output. For example, \texttt{colorlinks}, as its
name well implies, colors the links instead of using boxes; this is the
option used in this document.


\section{Package options}

All user-configurable aspects of \textsf{hyperref} are set using a
single `key=value' scheme (using the \textsf{keyval} package) with the
key \texttt{Hyp}. The options can be set either in the optional argument
to the \verb|\usepackage| command, or using the \verb|\hypersetup|
macro. When the package is loaded, a file \texttt{hyperref.cfg} is read
if it can be found, and this is a convenient place to set options on a
site-wide basis.

As an example, the behavior of a particular file could be controlled by:
\begin{itemize}

\item	a site-wide \texttt{hyperref.cfg} setting up the look of links,
adding backreferencing, and setting a PDF display default:

\begin{verbatim}
\hypersetup{backref,
pdfpagemode=FullScreen,
colorlinks=true}
\end{verbatim}

\item	A global option in the file, which is passed down to
\textsf{hyperref}:

\begin{verbatim}
\documentclass[dvips]{article}
\end{verbatim}

\item	File-specific options in the \verb|\usepackage| commands, which
override the ones set in \texttt{hyperref.cfg}:

\begin{verbatim}
\usepackage[pdftitle={A Perfect Day},colorlinks=false]{hyperref}
\end{verbatim}
\end{itemize}

Some options can be given at any time, but many are restricted: before
\verb|\begin{document}|, only in \verb|\usepackage[...]{hyperref}|,
before first use, etc.

In the key descriptions that follow, many options do not need a value,
as they default to the value true if used. These are the ones classed as
`boolean'. The values true and false can always be specified, however.

\subsection{General options}

Firstly, the options to specify general behavior and page size.

\medskip
\noindent\begin{longtable}{>{\ttfamily}ll>{\itshape}ll}
draft          & boolean & false & all hypertext options are turned off \\
final          & boolean & true  & all hypertext options are turned on \\
debug          & boolean & false & extra diagnostic messages are printed in \\
               &         &       & the log file \\
verbose        & boolean & false & same as debug \\
implicit       & boolean & true  & redefines \LaTeX\ internals \\
hypertexnames  & boolean & true  & use guessable names for links \\
naturalnames   & boolean & false & use \LaTeX-computed names for links \\
a4paper        & boolean & true  & sets paper size to 210mm $\times$ 297mm \\
a5paper        & boolean & false & sets paper size to 148mm $\times$ 210mm \\
b5paper        & boolean & false & sets paper size to 176mm $\times$ 250mm \\
letterpaper    & boolean & false & sets paper size to 8.5in $\times$ 11in \\
legalpaper     & boolean & false & sets paper size to 8.5in $\times$ 14in \\
executivepaper & boolean & false & sets paper size to 7.25in $\times$ 10.5in \\
setpagesize    & boolean & true  & sets page size by special driver commands
\end{longtable}

\subsection{Configuration options}

\noindent\begin{longtable}{>{\ttfamily}ll>{\itshape}lp{7cm}}
raiselinks & boolean & true  & In the hypertex driver, the height of links is normally calculcated by the driver as
                               simply the base line of contained text; this options forces \verb|\special| commands to
                               reflect the real height of the link (which could contain a graphic) \\
breaklinks & boolean & false & Allows link text to break across lines; since this cannot be accommodated in PDF, it is
                               only set true by default if the pdftex driver is used. This makes links on multiple lines
                               into different PDF links to the same target. \\
pageanchor & boolean & true  & Determines whether every page is given an implicit anchor at the top left corner. If this
                               is turned off, \verb|\printindex| will not contain
                               valid hyperlinks. \\
plainpages & boolean & false & Forces page anchors to be named by the arabic form of the page number, rather than the
                               formatted form. \\
nesting    & boolean & false & Allows links to be nested; no drivers currently support this.
\end{longtable}

Note for option \verb|breaklinks|:
The correct value is automatically set according to the driver features.
It can be overwritten for drivers that do not support broken links.
However, at any case, the link area will be wrong and displaced.

\subsection{Backend drivers}

If no driver is specified, the package defaults to loading the
\textsf{hypertex} driver.

\noindent\begin{longtable}{>{\ttfamily}lp{.8\hsize}}
dvipdfm     & Sets up \textsf{hyperref} for use with the \textsf{dvipdfm} driver.\\
dvipdfmx    & Sets up \textsf{hyperref} for use with the \textsf{dvipdfmx} driver.\\
dvips       & Sets up \textsf{hyperref} for use with the \textsf{dvips} driver. \\
dvipsone    & Sets up \textsf{hyperref} for use with the \textsf{dvipsone} driver. \\
dviwindo    & Sets up \textsf{hyperref} for use with the \textsf{dviwindo} Windows previewer. \\
hypertex    & Sets up \textsf{hyperref} for use with the Hyper\TeX-compliant drivers. \\
latex2html  & Redefines a few macros for compatibility with \textsf{latex2html}. \\
nativepdf   & An alias for \textsf{dvips} \\
pdfmark     & An alias for \textsf{dvips} \\
pdftex      & Sets up \textsf{hyperref} for use with the \textsf{pdftex} program.\\
ps2pdf      & Redefines a few macros for compatibility with
              Ghostscript's PDF writer, otherwise identical to
              \textsf{dvips}. \\
tex4ht      & For use with \textsf{\TeX4ht} \\
textures    & For use with \textsf{Textures} \\
vtex        & For use with MicroPress' \textsf{VTeX}; the PDF
                       and HTML backends are detected automatically. \\
vtexpdfmark & For use with \textsf{VTeX}'s PostScript backend. \\
xetex       & For use with Xe\TeX (using backend for dvipdfm).
\end{longtable}
\smallskip

If you use \textsf{dviwindo}, you may need to redefine the macro
\ci{wwwbrowser} (the default is \verb|C:\netscape\netscape|) to tell
\textsf{dviwindo} what program to launch. Thus, users of Internet
Explorer might add something like this to hyperref.cfg:

\begin{verbatim}
\renewcommand{wwwbrowser}{C:\string\Program\space
  Files\string\Plus!\string\Microsoft\space
  Internet\string\iexplore.exe}
\end{verbatim}

\subsection{Extension options}
\noindent\begin{longtable}{>{\ttfamily}ll>{\itshape}lp{6cm}}
extension      & text    &         & Set the file extension (e.g.\ \textsf{dvi}) which will be appended to file links
                                     created if you use the \textsf{xr} package. \\
hyperfigures   & boolean &         & \\
backref        & text    & false   & Adds `backlink' text to the end of each item in the bibliography, as a list of
                                     section numbers. This can only work properly \emph{if} there is a blank line after
                                     each \verb|\bibitem|. Supported values are \verb|section|, \verb|slide|, \verb|page|,
                                     \verb|none|, or \verb|false|. If no value is given, \verb|section| is taken as default.\\
pagebackref    & boolean & false   & Adds `backlink' text to the end of each item in the bibliography, as a list of page
                                     numbers. \\
hyperindex     & boolean & true    & Makes the page numbers of index entries into hyperlinks. Relays on unique
                                     page anchors (\verb|pageanchor|, \ldots)\\
                                     \verb|pageanchors| and \verb|plainpages=false|. \\
hyperfootnotes & boolean & true    & Makes the footnote marks into hyperlinks to the footnote text.
                                     Easily broken \ldots\\
encap          &         &         & Sets encap character for hyperindex \\
linktocpage    & boolean & false   & make page number, not text, be link on TOC, LOF and LOT \\
breaklinks     & boolean & false   & allow links to break over lines by making links over multiple lines into PDF links to
                                     the same target \\
colorlinks     & boolean & false   & Colors the text of links and anchors. The colors chosen depend on the the type of
                                     link. At present the only types of link distinguished are citations, page references,
                                     URLs, local file references, and other links. \\
linkcolor      & color   & red     & Color for normal internal links. \\
anchorcolor    & color   & black   & Color for anchor text. \\
citecolor      & color   & green   & Color for bibliographical citations in text. \\
filecolor      & color   & magenta & Color for URLs which open local files. \\
menucolor      & color   & red     & Color for Acrobat menu items. \\
runcolor       & color   & filecolor & Color for run links (launch annotations). \\
urlcolor       & color   & cyan    & Color for linked URLs. \\
frenchlinks    & boolean & false   & use small caps instead of color for links
\end{longtable} \smallskip

Note that all color names must be defined before use, following the
normal system of the standard \LaTeX\ \textsf{color} package.

\subsection{PDF-specific display options}
\noindent\begin{longtable}{@{}>{\ttfamily}ll>{\itshape}lp{7.5cm}@{}}
bookmarks          & boolean   & true   & A set of Acrobat bookmarks are written, in a manner similar to the
                                           table of contents, requiring two passes of \LaTeX. Some postprocessing
                                           of the bookmark file (file extension \texttt{.out}) may be needed to
                                           translate \LaTeX\ codes, since bookmarks must be written in  PDFEncoding.
                                           To aid this  process, the \texttt{.out} file is not rewritten by \LaTeX\
                                           if it is edited to contain a line \verb|\let\WriteBookmarks\relax| \\
bookmarksopen      & boolean   & false   & If Acrobat bookmarks are requested, show them with all the subtrees
                                           expanded. \\
bookmarksopenlevel & parameter &         & level (\ci{maxdimen}) to which bookmarks are open \\
bookmarksnumbered  & boolean   & false   & If Acrobat bookmarks are requested, include section numbers. \\
bookmarkstype      & text      & toc     & to specify which `toc' file to mimic \\
CJKbookmarks       & boolean   & false   &
    This option should be used to produce CJK bookmarks.
    Package \verb|hyperref|
    supports both normal and preprocessed mode of the CJK package;
    during the creation of bookmarks, it simply replaces CJK's macros
    with special versions which expand to the corresponding character
    codes.  Note that without the `unicode' option of hyperref you get
    PDF files which actually violate the PDF specification because
    non-Unicode character codes are used -- some PDF readers localized
    for CJK languages (most notably Acroread itself) support this.
    Also note that option `CJKbookmarks' cannot be used together
    with option `unicode'.

    No mechanism is provided to translate non-Unicode bookmarks to
    Unicode; for portable PDF documents only Unicode encoding should
    be used.\\
pdfhighlight       & name      & /I      & How link buttons behave when selected; /I is for inverse (the default);
                                           the other possibilities are /N (no effect), /O (outline), and /P (inset
                                           highlighting). \\
citebordercolor    & RGB color & 0 1 0   & The color of the box around citations \\
filebordercolor    & RGB color & 0 .5 .5 & The color of the box around links to files \\
linkbordercolor    & RGB color & 1 0 0   & The color of the box around normal links \\
menubordercolor    & RGB color & 1 0 0   & The color of the box around Acrobat menu links \\
urlbordercolor     & RGB color & 0 1 1   & The color of the box around links to URLs \\
runbordercolor     & RGB color & 0 .7 .7 & color of border around `run' links \\
pdfborder          &           & 0 0 1   & The style of box around links; defaults to a box with lines of 1pt thickness,
                                           but the colorlinks option resets it to produce no border.
\end{longtable}

Note that the color of link borders can be specified \emph{only} as 3
numbers in the range 0..1, giving an RGB color. You cannot use colors
defined in \TeX.

The bookmark commands are stored in a file called
\textit{jobname}\texttt{.out}. The files is not processed by \LaTeX\ so
any markup is passed through. You can postprocess this file as needed;
as an aid for this, the \texttt{.out} file is not overwritten on the
next \TeX\ run if it is edited to contain the line \\

\begin{verbatim}
\let\WriteBookmarks\relax
\end{verbatim}

\subsection{PDF display and information options}
\noindent\begin{longtable}{>{\ttfamily}ll>{\itshape}lp{6cm}}
baseurl            & URL     &       & Sets the base URL of the PDF document \\
pdfpagemode        & text    & empty & Determines how the file is opening in Acrobat; the possibilities are
                                       \texttt{UseNone}, \texttt{UseThumbs} (show thumbnails), \texttt{UseOutlines}
                                       (show bookmarks), \texttt{FullScreen}, \texttt{UseOC} (PDF 1.5),
                                       and \texttt{UseAttachments} (PDF 1.6). If no mode if explicitly chosen, but the
                                       bookmarks option is set, \texttt{UseOutlines} is used. \\
pdftitle           & text    &       & Sets the document information Title field \\
pdfauthor          & text    &       & Sets the document information Author field \\
pdfsubject         & text    &       & Sets the document information Subject field \\
pdfcreator         & text    &       & Sets the document information Creator field \\
pdfproducer        & text    &       & Sets the document information Producer field \\
pdfkeywords        & text    &       & Sets the document information Keywords field \\
pdfview            & text    & XYZ   & Sets the default PDF `view' for each link \\
pdfstartpage       & text    & 1     & Determines on which page the PDF file is opened. \\
pdfstartview       & text    & Fit   & Set the startup page view \\
pdfpagescrop       & n n n n &       & Sets the default PDF crop box for pages. This should be a set of four numbers \\
pdfcenterwindow    & boolean & false & position the document window in the center of the screen \\
pdfdirection       & text    & empty & direction setting \\
pdfdisplaydoctitle & boolean & false & display document title instead of
                                       file name in title bar\\
pdfduplex          & text    & empty & paper handling option for print dialog\\
pdffitwindow       & boolean & false & resize document window to fit document size \\
pdflang            & text    & empty & PDF language identifier (RFC 3066)\\
pdfmenubar         & boolean & true  & make PDF viewer's menu bar visible \\
pdfnewwindow       & boolean & false & make links that open another PDF file start a new window \\
pdfnonfullscreenpagemode
                   & boolean & empty & page mode setting on exiting
                                       full-screen mode\\
pdfnumcopies       & integer & empty & number of printed copies \\
pdfpagelayout      & text    & empty & set layout of PDF pages \\
pdfpagelabels      & boolean & true  & set PDF page labels \\
pdfpagetransition  & text    & empty & set PDF page transition style \\
pdfpicktrackbypdfsize & text & empty & set option for print dialog \\
pdfprintarea       & text    & empty & set /PrintArea of viewer preferences \\
pdfprintclip       & text    & empty & set /PrintClip of viewer preferences \\
pdfprintpagerange  & n n (n n)*
                             & empty & set /PrintPageRange of viewer
                                       preferences\\
pdfprintscaling    & text    & empty & page scaling option for print dialog;
                                       valid values are \texttt{None} and
                                       \texttt{AppDefault} \\
pdftoolbar         & boolean & true  & make PDF toolbar visible \\
pdfviewarea        & text    & empty & set /ViewArea of viewer preferences \\
pdfviewclip        & text    & empty & set /ViewClip of viewer preferences \\
pdfwindowui        & boolean & true  & make PDF user interface elements visible \\
unicode            & boolean & false & Unicode encoded PDF strings
\end{longtable}

Each link in Acrobat carries its own magnification level, which is set
using PDF coordinate space, which is not the same as \TeX's. The unit
is bp and the origin is in the lower left corner. See also
\verb|\hypercalcbp| that is explained on page \pageref{hypercalcbp}.
pdf\TeX\
works by supplying default values for \texttt{XYZ} (horizontal $\times$
vertical $\times$ zoom) and \texttt{FitBH}. However, drivers using
\texttt{pdfmark} do not supply defaults, so \textsf{hyperref} passes in
a value of -32768, which causes Acrobat to set (usually) sensible
defaults. The following are possible values for the \texttt{pdfview} and
\texttt{pdfstartview} parameters.

\noindent\begin{longtable}{>{\ttfamily}l>{\itshape}lp{7cm}}
XYZ   & left top zoom         & Sets a coordinate and a zoom factor. If any one is null, the source link value is used.
                                \textit{null null null} will give the same values as the current page. \\
Fit   &                       & Fits the page to the window. \\
FitH  & top                   & Fits the width of the page to the window. \\
FitV  & left                  & Fits the height of the page to the window. \\
FitR  & left bottom right top & Fits the rectangle specified by the four coordinates to the window. \\
FitB  &                       & Fits the page bounding box to the window. \\
FitBH & top                   & Fits the width of the page bounding box to the window. \\
FitBV & left                  & Fits the height of the page bounding box to the window. \\
\end{longtable}

The \texttt{pdfpagelayout} can be one of the following values.

\noindent\begin{longtable}{>{\ttfamily}lp{10cm}}
SinglePage     & Displays a single page; advancing flips the page \\
OneColumn      & Displays the document in one column; continuous scrolling. \\
TwoColumnLeft  & Displays the document in two columns, odd-numbered pages to the left. \\
TwoColumnRight & Displays the document in two columns, odd-numbered pages to the right.
\end{longtable}

Finally, the \texttt{pdfpagetransition} can be one of the following
values, where \textit{/Di} stands for direction of motion in degrees,
generally in 90$^{\circ}$\ steps, \textit{/Dm} is a horizontal
(\texttt{/H}) or vertical (\texttt{/V}) dimension (e.g.\ \texttt{Blinds
/Dm /V}), and \textit{/M} is for motion, either in (\texttt{/I}) or out
(\texttt{/O}).

\noindent\begin{longtable}{>{\ttfamily}llp{8.5cm}}
Blinds   & /Dm    & Multiple lines distributed evenly across the screen sweep in the same direction to reveal the new
                    page. \\
Box      & /M     & A box sweeps in or out. \\
Dissolve &        & The page image dissolves in a piecemeal fashion to reveal the new page. \\
Glitter  & /Di    & Similar to Dissolve, except the effect sweeps across the screen. \\
Split    & /Dm /M & Two lines sweep across the screen to reveal the new page. \\
Wipe     & /Di    & A single line sweeps across the screen to reveal the new page.
\end{longtable}

\subsection{Big alphabetical list}

The following is a complete listing of available options for
\textsf{hyperref}, arranged alphabetically.

\noindent\begin{longtable}{>{\ttfamily}llp{6cm}}
a4paper            &                        & use A4 paper \\
a5paper            &                        & use A5 paper \\
anchorcolor        & \textit{black}         & set color of anchors \\
b5paper            &                        & use B5 paper \\
backref            & \textit{false}         & do bibliographical back references \\
baseurl            & \textit{empty}         & set base URL for document \\
bookmarks          & \textit{true}          & make bookmarks \\
bookmarksnumbered  & \textit{false}         & put section numbers in bookmarks \\
bookmarksopen      & \textit{false}         & open up bookmark tree \\
bookmarksopenlevel & \ttfamily\ci{maxdimen} & level to which bookmarks are open \\
bookmarkstype      & \textit{toc}           & to specify which `toc' file to mimic \\
breaklinks         & \textit{false}         & allow links to break over lines \\
CJKbookmarks       & \textit{false}         & to produce CJK bookmarks\\
citebordercolor    & \textit{0 1 0}         & color of border around cites \\
citecolor          & \textit{green}         & color of citation links \\
colorlinks         & \textit{false}         & color links \\
                   & \textit{true}          & (\textsf{tex4ht}, \textsf{dviwindo}) \\
debug              & \textit{false}         & provide details of anchors defined; same as verbose \\
draft              & \textit{false}         & do not do any hyperlinking \\
dvipdfm            &                        & use \textsf{dvipdfm} backend \\
dvipdfmx           &                        & use \textsf{dvipdfmx} backend \\
dvips              &                        & use \textsf{dvips} backend \\
dvipsone           &                        & use \textsf{dvipsone} backend \\
dviwindo           &                        & use \textsf{dviwindo} backend \\
encap              &                        & to set encap character for hyperindex \\
executivepaper     &                        & use executivepaper \\
extension          & \textit{dvi}           & suffix of linked files \\
filebordercolor    & \textit{0 .5 .5}       & color of border around file links \\
filecolor          & \textit{cyan}          & color of file links \\
final              & \textit{true}          & opposite of option draft \\
frenchlinks        & \textit{false}         & use small caps instead of color for links \\
hyperfigures       & \textit{false}         & make figures hyper links \\
hyperfootnotes     & \textit{true}          & set up hyperlinked footnotes \\
hyperindex         & \textit{true}          & set up hyperlinked indices \\
hypertex           &                        & use \textsf{Hyper\TeX} backend \\
hypertexnames      & \textit{true}          & use guessable names for links \\
implicit           & \textit{true}          & redefine \LaTeX\ internals \\
latex2html         &                        & use \textsf{\LaTeX2HTML} backend \\
legalpaper         &                        & use legalpaper \\
letterpaper        &                        & use letterpaper \\
linkbordercolor    & \textit{1 0 0}         & color of border around links \\
linkcolor          & \textit{red}           & color of links \\
linktocpage        & \textit{false}         & make page number, not text, be link on TOC, LOF and LOT \\
menubordercolor    & \textit{1 0 0}         & color of border around menu links \\
menucolor          & \textit{red}           & color for menu links \\
nativepdf          & \textit{false}         & an alias for \textsf{dvips} \\
naturalnames       & \textit{false}         & use \LaTeX-computed names for links \\
nesting            & \textit{false}         & allow nesting of links \\
pageanchor         & \textit{true}          & put an anchor on every page \\
pagebackref        & \textit{false}         & backreference by page number \\
pdfauthor          & \textit{empty}         & text for PDF Author field \\
pdfborder          & \textit{0 0 1}         & width of PDF link border \\
                   & \textit{0 0 0}         & (\texttt{colorlinks)} \\
pdfcenterwindow    & \textit{false}         & position the document window in the center of the screen \\
pdfcreator         & \textit{LaTeX with}    & text for PDF Creator field \\
                   & \textit{hyperref}      & \\
                   & \textit{package}       & \\
pdfdirection       & \textit{empty}         & direction setting \\
pdfdisplaydoctitle & \textit{false}         & display document title instead
                                              of file name in title bar\\
pdfduplex          & \textit{empty}         & paper handling option for
                                              print dialog\\
pdffitwindow       & \textit{false}         & resize document window to fit document size \\
pdfhighlight       & \textit{/I}            & set highlighting of PDF links \\
pdfkeywords        & \textit{empty}         & text for PDF Keywords field \\
pdflang            & \textit{empty}         & PDF language identifier (RFC 3066) \\
pdfmark            & \textit{false}         & an alias for \textsf{dvips} \\
pdfmenubar         & \textit{true}          & make PDF viewer's menu bar visible \\
pdfnewwindow       & \textit{false}         & make links that open another PDF \\
                   &                        & file start a new window \\
pdfnonfullscreenpagemode
                   & \textit{empty}         & page mode setting on exiting
                                              full-screen mode\\
pdfnumcopies       & \textit{empty}         & number of printed copies\\
pdfpagelayout      & \textit{empty}         & set layout of PDF pages \\
pdfpagemode        & \textit{empty}         & set default mode of PDF display \\
pdfpagelabels      & \textit{true}          & set PDF page labels \\
pdfpagescrop       & \textit{empty}         & set crop size of PDF document \\
pdfpagetransition  & \textit{empty}         & set PDF page transition style \\
pdfpicktrackbypdfsize
                   & \textit{empty}         & set option for print dialog \\
pdfprintarea       & \textit{empty}         & set /PrintArea of viewer preferences \\
pdfprintclip       & \textit{empty}         & set /PrintClip of viewer preferences \\
pdfprintpagerange  & \textit{empty}         & set /PrintPageRange of viewer preferences \\
pdfprintscaling    & \textit{empty}         & page scaling option for print dialog \\
pdfproducer        & \textit{empty}         & text for PDF Producer field \\
pdfstartpage       & \textit{1}             & page at which PDF document opens \\
pdfstartview       & \textit{Fit}           & starting view of PDF document \\
pdfsubject         & \textit{empty}         & text for PDF Subject field \\
pdftex             &                        & use \textsf{pdf\TeX} backend \\
pdftitle           & \textit{empty}         & text for PDF Title field \\
pdftoolbar         & \textit{true}          & make PDF toolbar visible \\
pdfview            & \textit{XYZ}           & PDF `view' when on link traversal \\
pdfviewarea        & \textit{empty}         & set /ViewArea of viewer preferences \\
pdfviewclip        & \textit{empty}         & set /ViewClip of viewer preferences \\
pdfwindowui        & \textit{true}          & make PDF user interface elements visible \\
plainpages         & \textit{false}         & do page number anchors as plain arabic \\
ps2pdf             &                        & use \textsf{ps2pdf} backend \\
raiselinks         & \textit{false}         & raise up links (for \textsf{Hyper\TeX} backend) \\
runbordercolor     & \textit{0 .7 .7}       & color of border around `run' links \\
runcolor           & \textit{filecolor}     & color of `run' links\\
setpagesize        & \textit{true}          & set page size by special driver commands \\
tex4ht             &                        & use \textsf{\TeX4ht} backend \\
textures           &                        & use \textsf{Textures} backend \\
unicode            & \textit{false}         & Unicode encoded pdf strings \\
urlbordercolor     & \textit{0 1 1}         & color of border around URL links \\
urlcolor           & \textit{magenta}       & color of URL links \\
verbose            & \textit{false}         & be chatty \\
vtex               &                        & use \textsf{VTeX} backend\\
xetex              &                        & use \textsf{Xe\TeX} backend\\
\end{longtable}

\section{Additional user macros}

If you need to make references to URLs, or write explicit links, the
following low-level user macros are provided:

\begin{cmdsyntax}
\ci{href}\verb|{|\emph{URL}\verb|}{|\emph{text}\verb|}|
\end{cmdsyntax}

\noindent The \emph{text} is made a hyperlink to the \emph{URL}; this
must be a full URL (relative to the base URL, if that is defined). The
special characters \# and \~{} do \emph{not} need to be escaped in any
way.

\begin{cmdsyntax}
\ci{url}\verb|{|\emph{URL}\verb|}|
\end{cmdsyntax}

\noindent Similar to
\ci{href}\verb|{|\emph{URL}\verb|}{|\ci{nolinkurl}\verb|{|\emph{URL}\verb|}}|.
Depending on the driver \verb|\href| also tries to detect the link type.
Thus the result can be a url link, file link, \dots

\begin{cmdsyntax}
\ci{nolinkurl}\verb|{|\emph{URL}\verb|}|
\end{cmdsyntax}

\noindent Write \emph{URL} in the same way as \verb|\url|,
  without creating a hyperlink.

\begin{cmdsyntax}
\ci{hyperbaseurl}\verb|{|\emph{URL}\verb|}|
\end{cmdsyntax}

\noindent A base \emph{URL} is established, which is prepended to other
specified URLs, to make it easier to write portable documents.

\begin{cmdsyntax}
\ci{hyperimage}\verb|{|\emph{imageURL}\verb|}{|\emph{text}\verb|}|
\end{cmdsyntax}

\noindent The link to the image referenced by the URL is inserted, using
\emph{text} as the anchor.

  For drivers that produce HTML, the image itself is inserted by the
browser, with the \emph{text} being ignored completely.

\begin{cmdsyntax}
\ci{hyperdef}\verb|{|\emph{category}\verb|}{|\emph{name}\verb|}{|\emph{text}\verb|}|
\end{cmdsyntax}

\noindent A target area of the document (the \emph{text}) is marked, and
given the name \emph{category.name}

\begin{cmdsyntax}
\ci{hyperref}\verb|{|\emph{URL}\verb|}{|\emph{category}\verb|}{|\emph{name}\verb|}{|\emph{text}\verb|}|
\end{cmdsyntax}

\noindent \emph{text} is made into a link to \emph{URL\#category.name}

\begin{cmdsyntax}
\ci{hyperref}\verb|[|\emph{label}\verb|]{|\emph{text}\verb|}|
\end{cmdsyntax}

\noindent
\emph{text} is made into a link to the same place as
\verb|\ref{|\emph{label}\verb|}| would be linked.

\begin{cmdsyntax}
\ci{hyperlink}\verb|{|\emph{name}\verb|}{|\emph{text}\verb|}|
\end{cmdsyntax}
\begin{cmdsyntax}
\ci{hypertarget}\verb|{|\emph{name}\verb|}{|\emph{text}\verb|}|
\end{cmdsyntax}

\noindent A simple internal link is created with \verb|\hypertarget|,
with two parameters of an anchor \emph{name}, and anchor
\emph{text}. \verb|\hyperlink| has two arguments, the name of a
hypertext object defined somewhere by \verb|\hypertarget|, and the
\emph{text} which be used as the link on the page.

Note that in HTML parlance, the \verb|\hyperlink| command inserts a
notional \# in front of each link, making it relative to the current
testdocument; \verb|\href| expects a full URL.

\begin{cmdsyntax}
\ci{phantomsection}
\end{cmdsyntax}

\noindent
This sets an anchor at this location. It works similar to
\verb|\hypertarget{}{}| with an automatically choosen anchor name.
Often it is used in conjunction
with \verb|\addcontentsline| for sectionlike things (index, bibliography,
preface). \verb|\addcontentsline| refers to the latest previous location
where an anchor is set. Example:
\begin{quote}
\begin{verbatim}
\cleardoublepage
\phantomsection
\addcontentsline{toc}{chapter}{\indexname}
\printindex
\end{verbatim}
\end{quote}
Now the entry in the table of contents (and bookmarks) for the
index points to the start of the index page, not to a location
before this page.

\begin{cmdsyntax}
\ci{autoref}\verb|{|\emph{label}\verb|}|
\end{cmdsyntax}

\noindent
This is a replacement for the usual \verb|\ref| command that places a
contextual label in front of the reference. This gives your users a
bigger target to click for hyperlinks (e.g.\ `section 2' instead of
merely the number `2').

The label is worked out from the context of the original \verb|\label|
command by \textsf{hyperref} by using the macros listed below (shown
with their default values). The macros can be (re)defined in documents
using \verb|\(re)newcommand|; note that some of these macros are already
defined in the standard document classes. The mixture of lowercase and
uppercase initial letters is deliberate and corresponds to the author's
practice.

For each macro below, \textsf{hyperref} checks \ci{*autorefname} before
\ci{*name}.  For instance, it looks for \ci{figureautorefname} before
\ci{figurename}.

\noindent\begin{longtable}{lp{10cm}}
\textit{Macro}         & \textit{Default} \\
\ci{figurename}        & Figure \\
\ci{tablename}         & Table \\
\ci{partname}          & Part \\
\ci{appendixname}      & Appendix \\
\ci{equationname}      & Equation \\
\ci{Itemname}          & item \\
\ci{chaptername}       & chapter \\
\ci{sectionname}       & section \\
\ci{subsectionname}    & subsection \\
\ci{subsubsectionname} & subsubsection \\
\ci{paragraphname}     & paragraph \\
\ci{Hfootnotename}     & footnote \\
\ci{AMSname}           & Equation \\
\ci{theoremname}       & Theorem\\
\ci{page}              & page\\
\end{longtable}

Example for a redefinition if \textsf{babel} is used:
\begin{quote}
\begin{verbatim}
\usepackage[ngerman]{babel}
\addto\extrasngerman{%
  \def\subsectionautorefname{Unterkapitel}%
}
\end{verbatim}
\end{quote}

\begin{cmdsyntax}
\ci{autopageref}\verb|{|\emph{label}\verb|}|
\end{cmdsyntax}

\noindent
It replaces \verb|\pageref| and adds the name for page in front of
the page reference. First \ci{pageautorefname} is checked before
\ci{pagename}.

For instances where you want a reference to use the correct counter, but
not to create a link, there are starred forms:

\begin{cmdsyntax}
\ci{ref*}\verb|{|\emph{label}\verb|}|
\end{cmdsyntax}

\begin{cmdsyntax}
\ci{pageref*}\verb|{|\emph{label}\verb|}|
\end{cmdsyntax}

\begin{cmdsyntax}
\ci{autoref*}\verb|{|\emph{label}\verb|}|
\end{cmdsyntax}

\begin{cmdsyntax}
\ci{autopageref*}\verb|{|\emph{label}\verb|}|
\end{cmdsyntax}

A typical use would be to write
\begin{verbatim}
\hyperref[other]{that nice section (\ref*{other}) we read before}
\end{verbatim}

We want \verb|\ref*{other}| to generate the correct number, but not to
form a link, since we do this ourselves with \ci{hyperref}.

\begin{cmdsyntax}
\ci{pdfstringdef}\verb|{|\emph{macroname}\verb|}{|\emph{\TeX string}\verb|}|
\end{cmdsyntax}

\ci{pdfstringdef} returns a macro containing the PDF string. (Currently
this is done globally, but do not rely on it.) All the following tasks,
definitions and redefinitions are made in a group to keep them local:

\begin{itemize}
\item Switching to PD1 or PU encoding
\item Defining the \Quote{octal sequence commands} (\verb|\345|): \verb|\edef\3{\string\3}|
\item Special glyphs of \TeX: \verb|\{|, \verb|\%|, \verb|\&|, \verb|\space|, \verb|\dots|, etc.
\item National glyphs (\textsf{german.sty}, \textsf{french.sty}, etc.)
\item Logos: \verb|\TeX|, \verb|\eTeX|, \verb|\MF|, etc.
\item Disabling commands that do not provide useful functionality in bookmarks:
\verb|\label|, \verb|\index|, \verb|\glossary|, \verb|\discretionary|, \verb|\def|, \verb|\let|, etc.
\item \LaTeX's font commands like \verb|\textbf|, etc.
\item Support for \verb|\xspace| provided by the \textsf{xspace} package
\end{itemize}

In addition, parentheses are protected to avoid the danger of unsafe
unbalanced parentheses in the PDF string. For further details, see Heiko
Oberdiek's Euro\TeX\ paper distributed with \textsf{hyperref}.


\subsection{Replacement macros}

\textsf{hyperref} takes the text for bookmarks from the arguments of
commands like \ci{section}, which can contain things like math, colors,
or font changes, none of which will display in bookmarks as is.

\begin{cmdsyntax}
\ci{texorpdfstring}\verb|{|\emph{\TeX string}\verb|}{|\emph{PDFstring}\verb|}|
\end{cmdsyntax}

For example,
\begin{verbatim}
\section{Pythagoras:
  \texorpdfstring{$ a^2 + b^2 = c^2 $}{%
    a\texttwosuperior\ + b\texttwosuperior\ =
    c\texttwosuperior
  }%
}
\section{\texorpdfstring{\textcolor{red}}{}{Red} Mars}
\end{verbatim}

\ci{pdfstringdef} executes the hook \pdfstringdefPreHook before it
expands the string. Therefore, you can use this hook to perform
additional tasks or to disable additional commands.

\begin{verbatim}
\expandafter\def\expandafter\pdfstringdefPreHook
\expandafter{%
  \pdfstringdefPreHook
  \renewcommand{\mycommand}[1]{}%
}
\end{verbatim}

However, for disabling commands, an easier way is via
\ci{pdfstringdefDisableCommands}, which adds its argument to the
definition of \ci{pdfstringdefPreHook} (`@' can here be used as letter in
command names):

\begin{verbatim}
\pdfstringdefDisableCommands{%
  \let~\textasciitilde
  \def\url{\pdfstringdefwarn\url}%
  \let\textcolor\@gobble
}
\end{verbatim}

\subsection{Utility macros}

\label{hypercalcbp}
\begin{cmdsyntax}
\ci{hypercalcbp}\verb|{|\emph{dimen specification}\verb|}|
\end{cmdsyntax}
\noindent
\verb|\hypercalcbp| takes a \TeX\ dimen specification and
converts it to bp and returns the number without the unit.
This is useful for options \verb|pdfstartview| and \verb|pdfview|.
Example:
\begin{quote}
\begin{verbatim}
\hypersetup{
  pdfstartview={FitBH \hypercalcbp{\paperheight-\topmargin-1in
    -\headheight-\headsep}
}
\end{verbatim}
\end{quote}
The origin of the PDF coordinate system is the lower left corner.

Note, for calculations you need either package |calc| or
\eTeX. Nowadays the latter should automatically be enabled
for \LaTeX\ formats. Users without \eTeX, please, look
in the source documentation \verb|hyperref.dtx| for further
limitations.

Also \verb|\hypercalcbp| cannot be used in option specifications
of \verb|\documentclass| and \verb|\usepackage|, because
\LaTeX\ expands the option lists of these commands. However
package \verb|hyperref| is not yet loaded and an undefined control
sequence error would arise.

\section{Acrobat-specific behavior}
If you want to access the menu options of Acrobat Reader or Exchange, the following
macro is provided in the appropriate drivers:

\begin{cmdsyntax}
\ci{Acrobatmenu}\verb|{|\emph{menuoption}\verb|}{|\emph{text}\verb|}|
\end{cmdsyntax}

\noindent The \emph{text} is used to create a button which activates the appropriate \emph{menuoption}. The following table lists the option names you can use---comparison of this with the menus in Acrobat Reader or Exchange will show what they do. Obviously some are only appropriate to Exchange.

\medskip
\noindent\begin{longtable}{lp{9cm}}
File                          & Open, Close, Scan, Save, SaveAs, Optimizer:SaveAsOpt, Print, PageSetup, Quit \\
File$\rightarrow$Import       & ImportImage, ImportNotes, AcroForm:ImportFDF \\
File$\rightarrow$Export       & ExportNotes, AcroForm:ExportFDF \\
File$\rightarrow$DocumentInfo & GeneralInfo, OpenInfo, FontsInfo, SecurityInfo, Weblink:Base, AutoIndex:DocInfo \\
File$\rightarrow$Preferences  & GeneralPrefs, NotePrefs, FullScreenPrefs, Weblink:Prefs, AcroSearch:Preferences(Windows)
                                or, AcroSearch:Prefs(Mac), Cpt:Capture \\
Edit                          & Undo, Cut, Copy, Paste, Clear, SelectAll, Ole:CopyFile, TouchUp:TextAttributes,
                                TouchUp:FitTextToSelection, TouchUp:ShowLineMarkers, TouchUp:ShowCaptureSuspects,
                                TouchUp:FindSuspect, \\
                              & Properties \\
Edit$\rightarrow$Fields       & AcroForm:Duplicate, AcroForm:TabOrder \\
Document                      & Cpt:CapturePages, AcroForm:Actions, CropPages, RotatePages, InsertPages, ExtractPages,
                                ReplacePages, DeletePages, NewBookmark, SetBookmarkDest, CreateAllThumbs,
                                DeleteAllThumbs \\
View                          & ActualSize, FitVisible, FitWidth, FitPage, ZoomTo, FullScreen, FirstPage, PrevPage,
                                NextPage, LastPage, GoToPage, GoBack, GoForward, SinglePage, OneColumn, TwoColumns,
                                ArticleThreads, PageOnly, ShowBookmarks, ShowThumbs \\
Tools                         & Hand, ZoomIn, ZoomOut, SelectText, SelectGraphics, Note, Link, Thread, AcroForm:Tool,
                                Acro\_Movie:MoviePlayer, TouchUp:TextTool, Find, FindAgain, FindNextNote,
                                CreateNotesFile \\
Tools$\rightarrow$Search      & AcroSrch:Query, AcroSrch:Indexes, AcroSrch:Results, AcroSrch:Assist, AcroSrch:PrevDoc,
                                AcroSrch:PrevHit, AcroSrch:NextHit, AcroSrch:NextDoc \\
Window                        & ShowHideToolBar, ShowHideMenuBar, ShowHideClipboard, Cascade, TileHorizontal,
                                TileVertical, CloseAll \\
Help                          & HelpUserGuide, HelpTutorial, HelpExchange, HelpScan, HelpCapture, HelpPDFWriter,
                                HelpDistiller, HelpSearch, HelpCatalog, HelpReader, Weblink:Home \\
Help(Windows)                 & About
\end{longtable}

\section{PDF and HTML forms}
You must put your fields inside a \texttt{Form} environment (only one per file).

There are six macros to prepare fields:

\begin{cmdsyntax}
\ci{TextField}\verb|[|\emph{parameters}\verb|]{|\emph{label}\verb|}|
\end{cmdsyntax}

\begin{cmdsyntax}
\ci{CheckBox}\verb|[|\emph{parameters}\verb|]{|\emph{label}\verb|}|
\end{cmdsyntax}

\begin{cmdsyntax}
\ci{ChoiceMenu}\verb|[|\emph{parameters}\verb|]{|\emph{label}\verb|}{|\emph{choices}\verb|}|
\end{cmdsyntax}

\begin{cmdsyntax}
\ci{PushButton}\verb|[|\emph{parameters}\verb|]{|\emph{label}\verb|}|
\end{cmdsyntax}

\begin{cmdsyntax}
\ci{Submit}\verb|[|\emph{parameters}\verb|]{|\emph{label}\verb|}|
\end{cmdsyntax}

\begin{cmdsyntax}
\ci{Reset}\verb|[|\emph{parameters}\verb|]{|\emph{label}\verb|}|
\end{cmdsyntax}

The way forms and their labels are laid out is determined by:
\begin{cmdsyntax}
\ci{LayoutTextField}\verb|{|\emph{label}\verb|}{|\emph{field}\verb|}|
\end{cmdsyntax}

\begin{cmdsyntax}
\ci{LayoutChoiceField}\verb|{|\emph{label}\verb|}{|\emph{field}\verb|}|
\end{cmdsyntax}

\begin{cmdsyntax}
\ci{LayoutCheckField}\verb|{|\emph{label}\verb|}{|\emph{field}\verb|}|
\end{cmdsyntax}

These macros default to \#1 \#2

What is actually shown in as the field is determined by:
\begin{cmdsyntax}
\ci{MakeRadioField}\verb|{|\emph{width}\verb|}{|\emph{height}\verb|}|
\end{cmdsyntax}

\begin{cmdsyntax}
\ci{MakeCheckField}\verb|{|\emph{width}\verb|}{|\emph{height}\verb|}|
\end{cmdsyntax}
\begin{cmdsyntax}
\ci{MakeTextField}\verb|{|\emph{width}\verb|}{|\emph{height}\verb|}|
\end{cmdsyntax}
\begin{cmdsyntax}
\ci{MakeChoiceField}\verb|{|\emph{width}\verb|}{|\emph{height}\verb|}|
\end{cmdsyntax}

\begin{cmdsyntax}
\ci{MakeButtonField}\verb|{|\emph{text}\verb|}|
\end{cmdsyntax}

These macros default to \verb|\vbox to #2{\hbox to #1{\hfill}\vfill}|, except the
last, which defaults to \#1; it is used for buttons, and the special \ci{Submit} and \ci{Reset}
macros.

You may also want to redefine the following macros:
\begin{verbatim}
\def\DefaultHeightofSubmit{12pt}
\def\DefaultWidthofSubmit{2cm}
\def\DefaultHeightofReset{12pt}
\def\DefaultWidthofReset{2cm}
\def\DefaultHeightofCheckBox{0.8\baselineskip}
\def\DefaultWidthofCheckBox{0.8\baselineskip}
\def\DefaultHeightofChoiceMenu{0.8\baselineskip}
\def\DefaultWidthofChoiceMenu{0.8\baselineskip}
\def\DefaultHeightofText{\baselineskip}
\def\DefaultHeightofTextMultiline{4\baselineskip}
\def\DefaultWidthofText{3cm}
\end{verbatim}

\subsection{Forms environment parameters}

\smallskip\noindent\begin{longtable}{>{\ttfamily}l>{\itshape}lp{9cm}}
action   & URL  & The URL that will receive the form data if a \textsf{Submit} button is included in the form \\
encoding & name & The encoding for the string set to the URL; FDF-encoding is usual, and \texttt{html} is the only
                  valid value \\
method   & name & Used only when generating HTML; values can be \texttt{post} or \texttt{get} \\
\end{longtable}

\subsection{Forms optional parameters}
Note that all colors must be expressed as RGB triples, in the range 0..1 (i.e.\ \texttt{color=0 0
0.5})

\smallskip\noindent\begin{longtable}{>{\ttfamily}ll>{\itshape}ll}
accesskey       & key     &       & (as per HTML) \\
align           & number  & 0     & alignment within text field; 0 is left-aligned, \\
                &         &       & 1 is centered, 2 is right-aligned. \\
backgroundcolor &         &       & color of box \\
bordercolor     &         &       & color of border \\
bordersep       &         &       & box border gap \\
borderwidth     &         &       & width of box border \\
calculate       &         &       & JavaScript code to calculate the value of the field \\
charsize        & dimen   &       & font size of field text \\
checked         & boolean & false & whether option selected by default \\
color           &         &       & color of text in box \\
combo           & boolean & false & choice list is `combo' style \\
default         &         &       & default value \\
disabled        & boolean & false & field disabled \\
format          &         &       & JavaScript code to format the field \\
height          & dimen   &       & height of field box \\
hidden          & boolean & false & field hidden \\
ketstroke       &         &       & JavaScript code to control the keystrokes on entry \\
maxlen          & number  & 0     & number of characters allowed in text field \\
menulength      & number  & 4     & number of elements shown in list \\
multiline       & boolean & false & whether text box is multiline \\
name            & name    &       & name of field (defaults to label) \\
onblur          &         &       & JavaScript code \\
onchange        &         &       & JavaScript code \\
onclick         &         &       & JavaScript code \\
ondblclick      &         &       & JavaScript code \\
onfocus         &         &       & JavaScript code \\
onkeydown       &         &       & JavaScript code \\
onkeypress      &         &       & JavaScript code \\
onkeyup         &         &       & JavaScript code \\
onmousedown     &         &       & JavaScript code \\
onmousemove     &         &       & JavaScript code \\
onmouseout      &         &       & JavaScript code \\
onmouseover     &         &       & JavaScript code \\
onmouseup       &         &       & JavaScript code \\
onselect        &         &       & JavaScript code \\
password        & boolean & false & text field is `password' style \\
popdown         & boolean & false & choice list is `popdown' style \\
radio           & boolean & false & choice list is `radio' style \\
readonly        & boolean & false & field is readonly \\
rotation        & number  & 0     & rotation of the widget annotation
                                    (degree, counterclockwise, multiple of 90)\\
tabkey          &         &       & (as per HTML) \\
validate        &         &       & JavaScript code to validate the entry \\
value           &         &       & initial value \\
width           & dimen   &       & width of field box
\end{longtable}

\section{Defining a new driver}
A hyperref driver has to provide definitions for eight macros:

\smallskip
\noindent 1. \verb|\hyper@anchor|

\noindent 2. \verb|\hyper@link|

\noindent 3. \verb|\hyper@linkfile|

\noindent 4. \verb|\hyper@linkurl|

\noindent 5. \verb|\hyper@anchorstart|

\noindent 6. \verb|\hyper@anchorend|

\noindent 7. \verb|\hyper@linkstart|

\noindent 8. \verb|\hyper@linkend|
\smallskip

The draft option defines the macros as follows
\qquad\begin{verbatim}
\let\hyper@@anchor\@gobble
\gdef\hyper@link##1##2##3{##3}%
\def\hyper@linkurl##1##2{##1}%
\def\hyper@linkfile##1##2##3{##1}%
\let\hyper@anchorstart\@gobble
\let\hyper@anchorend\@empty
\let\hyper@linkstart\@gobbletwo
\let\hyper@linkend\@empty
\end{verbatim}

\section{Special support for other packages}

\textsf{hyperref} aims to cooperate with other packages, but there are
several possible sources for conflict, such as

\begin{itemize}

\item Packages that manipulate the bibliographic mechanism. Peter
William's \textsf{harvard} package is supported. However, the
recommended package is Patrick Daly's \textsf{natbib} package that has
specific \textsf{hyperref} hooks to allow reliable interaction. This
package covers a very wide variety of layouts and citation styles, all
of which work with \textsf{hyperref}.

\item Packages that typeset the contents of the \ci{label} and \ci{ref}
macros, such as \textsf{showkeys}. Since the \textsf{hyperref} package
redefines these commands, you must set \texttt{implicit=false} for these
packages to work.

\item Packages that do anything serious with the index.
\end{itemize}

The \textsf{hyperref} package is distributed with variants on two useful
packages designed to work especially well with it. These are \textsf{xr}
and \textsf{minitoc}, which support crossdocument links using \LaTeX's
normal \verb|\label/\ref| mechanisms and per-chapter tables of contents,
respectively.

\section{History and acknowledgments}

The original authors of \textsf{hyperbasics.tex} and
\textsf{hypertex.sty}, from which this package descends, are Tanmoy
Bhattacharya (\texttt{tanmoy@qcd.lanl.gov}) and Thorsten Ohl
\linebreak(\texttt{thorsten.ohl@physik.th-darmstadt.de}). \textsf{hyperref}
started as a simple port of their work to \LaTeXe\ standards, but
eventually I rewrote nearly everything, because I didn't understand a
lot of the original, and was only interested in getting it to work with
\LaTeX. I would like to thank Arthur Smith, Tanmoy Bhattacharya, Mark
Doyle, Paul Ginsparg, David Carlisle, T.\ V.\ Raman and Leslie Lamport
for comments, requests, thoughts and code to get the package into its
first useable state. Various other people are mentioned at the point in
the source where I had to change the code in later versions because of
problems they found.

Tanmoy found a great many of the bugs, and (even better) often provided
fixes, which has made the package more robust. The days spent on
Rev\TeX\ are entirely due to him! The investigations of Bill Moss
(\texttt{bmoss@math.clemson.edu}) into the later versions including
native PDF support uncovered a good many bugs, and his testing is
appreciated. Hans Hagen (\texttt{pragma@pi.net}) provided a lot of
insight into PDF.

Berthold Horn provided help, encouragement and sponsorship for the
\textsf{dvipsone} and \textsf{dviwindo} drivers. Sergey Lesenko provided
the changes needed for \textsf{dvipdf}, and \Hanh{} supplied all the
information needed for \textsf{pdftex}. Patrick Daly kindly updated his
\textsf{natbib} package to allow easy integration with
\textsf{hyperref}. Michael Mehlich's \textsf{hyper} package (developed
in parallel with \textsf{hyperref}) showed me solutions for some
problems. Hopefully the two packages will combine one day.

The forms creation section owes a great deal to: T.\ V.\ Raman, for
encouragement, support and ideas; Thomas Merz, whose book \emph{Web
Publishing with Acrobat/PDF} provided crucial insights; D.\ P.\ Story,
whose detailed article about pdfmarks and forms solved many practical
problems; and Hans Hagen, who explained how to do it in \textsf{pdftex}.

Steve Dandy recreated the manual source in July 2003 after it had been
lost.

Especial extra thanks to David Carlisle for the \textsf{backref} module,
the ps2pdf and dviwindo support, frequent general rewrites of my bad
code, and for working on changes to the \textsf{xr} package to suit
\textsf{hyperref}.

\begingroup
  \makeatletter
  \let\chapter=\section
  % subsections goes into bookmarks but not toc
  \hypersetup{bookmarksopenlevel=1}
  \addtocontents{toc}{\protect\setcounter{tocdepth}{1}}
  % The \section command acts as \subsection.
  % Additionally the title is converted to lowercase except
  % for the first letter.
  \def\section{%
    \let\section\lc@subsection
    \lc@subsection
  }
  \def\lc@subsection{%
    \@ifstar{\def\mystar{*}\lc@sec}%
            {\let\mystar\@empty\lc@sec}%
  }
  \def\lc@sec#1{%
    \lc@@sec#1\@nil
  }
  \def\lc@@sec#1#2\@nil{%
    \begingroup
      \def\a{#1}%
      \lowercase{%
        \edef\x{\endgroup
          \noexpand\subsection\mystar{\a#2}%
        }%
      }%
    \x
  }
  \include{fdl}
\endgroup

\end{document}
