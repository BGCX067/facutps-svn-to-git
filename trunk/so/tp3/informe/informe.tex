\documentclass[a4paper, 10pt]{article}
\usepackage[spanish] {babel}
\title{Taller 1}
\usepackage{caratula}
\usepackage[pdftex]{graphicx}
\usepackage{amssymb}

\setlength{\leftmargin}{2cm}
\setlength{\rightmargin}{2cm}
\setlength{\oddsidemargin}{-1cm}
\setlength{\evensidemargin}{-1cm}
\setlength{\topmargin}{-1cm}
\setlength{\textwidth}{18cm}
\setlength{\textheight}{25cm}


\usepackage{fancyhdr}
\usepackage{listings}
\pagestyle{fancy}
\fancyhf{}
\fancyhead [L]{\scriptsize Trabajo Pr\'actico N$^{\circ}$3}
\fancyhead [R]{\scriptsize Mancuso, Mataloni}%1{20pt}
\fancyfoot[C]{\thepage}
\renewcommand{\footrulewidth}{0.4pt}

\begin{document}
\materia{Sistemas Operativos}
\submateria{Segundo Cuatrimestre de 2010}
\titulo{Trabajo Pr\'actico N$^{\circ}$3}
\subtitulo{Algoritmos en sistemas distribuidos}
\grupo{Grupo}
\integrante{Mataloni Alejandro}{706/07}{amataloni@gmail.com}
\integrante{Mancuso Emiliano}{597/07}{emiliano.mancuso@gmail.com}
\maketitle

\newpage

\section{Ejercicio 1}

En este trabajo implementaremos un algoritmo para resolver el problema de exclusi\'on mutua en sistemas distribuidos. 
El prototipo de ejercicio consta de un cliente desarrollado por la c/'atedra y un servidor el cual tenemos que implementar basandonos en el algoritmo de Ricart-Agrawala.
\newline

Para realizar esta tarea, nos ocuparemos de manejar los mensajes enviados cliente-servidor, as\'i como tambi\'en con los servidor-servidor.
Cada cliente tiene asignado un servidor y solo puede enviar y recibir mensajes del mismo. Sin embargo un servidor se comunica con su \'unico cliente pero tambi\'en con el resto de los servidores.

El servidor b\'asicamente se encuentra recibiendo mensajes de cualquier tipo. Cuenta con una serie de variables \'utiles a para definir su comportamiento:

\begin{itemize}
\item int seqnrorecibida;
\item int ourSeqNumber;
\item int highestSeqNum;
\item int outstandingReplycount;
\item bool requestingCriticalSection;
\item bool replyDeferred[cantserv];
\item bool deferIt;
\end{itemize}

\vspace*{1em}
Hay 2 grandes tipos de mensajes:

\begin{itemize}
\item cliente-servidor
\item servidor-servidor
\end{itemize}

\subsection{Mensajes cliente-servidor}
Mensaje tipo \textbf{PEDIDO}: 
\begin{itemize}
\item El cliente le pide al servidor, acceso a la secci\'on cr\'itica.
\item El servidor le responde al cliente cuando puede acceder a su secci\'on cr\'itica. Tambi\'en envia un mensaje de tipo \textbf{SERVPED} al resto de los servidores.
\end{itemize}

Mensaje tipo \textbf{LIBERO}: 
\begin{itemize}
\item El cliente le avisa al servidor, que sale de la secci\'on cr\'itica.
\item El servidor asociado, responde a los pedidos de los servidores pendientes enviando un mensaje de tipo \textbf{SERVRES}.
\end{itemize}

Mensaje tipo \textbf{TERMINO}: 
\begin{itemize}
\item El cliente le avisa al servidor, que termina su ejecuci\'on.
\end{itemize}

\newpage

\subsection{Mensajes servidor-servidor}
Mensaje tipo \textbf{SERVPED}: 
\begin{itemize}
\item Al enviar, se pide permiso para acceder a la secci\'on cr\'itica.
\item Al recibir, se le responde inmediatamente con un mensaje de tipo \textbf{SERVRES}, o en caso de tener prioridad sobre el otro servidor, se pospone la respuesta.
\end{itemize}

Mensaje tipo \textbf{SERVRES}:
\begin{itemize}
\item Al enviar, el servidor responde al pedido de la secci\'on cr\'itica.
\item Al recibir, si se tienen las respuestas de todos los servidores, se le da el permiso al cliente asociado para entrar a la secci\'on cr\'itica a trav\'es de un mensaje de tipo \textvf{PEDIDO}.
\end{itemize}
\vspace*{4em}

\section{Pruebas realizadas}

M\'as all\'a de hacer muchos \textit{printf} para ver que nuestro sistema estaba haciendo lo que deber\'ia, nosotros quis\'imos asegurarnos de que funcionaba correctamente.
Hicimos lo siguiente, creamos una prueba propia en donde 2 clientes se disputar\'ian la secci\'on cr\'itica.
Un cliente empieza pidiendo acceder a la secci\'on cr\'itica y no liberarla por un largo rato, en el medio el otro cliente hace el pedido. Los archivos que escribimos fueron los siguientes:

\vspace*{2em}
\lstset{language=[ANSI]C++}
\begin{lstlisting}
 Cliente 1 		Cliente 2
 pedir			nada
 nada			pedir
 nada			escribir Yo tengo el poder!
 nada			liberar
 nada
 nada
 nada
 liberar	
\end{lstlisting}
\vspace*{2em}


Aparte de esto tambi\'en modificamos un poco el c\'odigo del cliente. Lo que hicimos fue que cuando se lee una instrucci\'on de \textit{nada} el cliente se mande a dormir un tiempo. Esto es para que el cliente que primero accede a la secci\'on cr\'itica luego se quede haciendo \textit{sleep} en las siguientes operaciones as\'i el otro cliente pide la secci\'on cr\'itica y se queda esperando hasta que el primero no termine de hacer todos los \textit{nada} que tiene.

\end{document}